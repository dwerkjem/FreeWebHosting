\section{Setting up files and services}

The first step in creating a free website is to set up the necessary files and services. This section will guide you through the process of preparing your development environment, selecting a hosting service, and organizing your project files.

\subsection{Choosing a Hosting Service}

\subsubsection{What is a Hosting Service?}
A hosting service is a company that provides the infrastructure and technology needed to make your website accessible on the internet. They store your website's files on their servers and ensure that they are available to users who visit your site. Hosting services can vary widely in terms of features, performance, and cost. It is possible to host a website yourself, but for most users, especially those new to web development, using a hosting service is the most practical option. There are several free hosting services available that allow you to host your website without any cost. Some popular options include:
\begin{itemize}
    \item \textbf{GitHub Pages:} Ideal for static websites, it allows you to host your site directly from a GitHub repository.
    \item \textbf{Netlify:} Offers free hosting with continuous deployment from Git repositories, making it easy to update your site.
    \item \textbf{Vercel:} Similar to Netlify, it provides free hosting with a focus on performance and ease of use.
    \item \textbf{Firebase Hosting:} Great for dynamic web applications, it offers free hosting with a generous quota.
    \item \textbf{Surge:} A simple command-line tool for publishing static sites, ideal for quick deployments.
\end{itemize}

For this guide, we will focus on using GitHub Pages as it is widely used and integrates well with version control. However, the principles outlined here can be applied to any of the mentioned services. Make sure to sign up for an account with your chosen hosting service before proceeding.

\subsubsection{GitHub Pages}
GitHub Pages is a free hosting service provided by GitHub that allows you to host static websites directly from a GitHub repository. It is designed to work seamlessly with Git, making it easy to deploy your site by simply pushing changes to your repository. GitHub Pages is an excellent choice for personal projects, portfolios, and documentation sites, as it supports custom domains and HTTPS out of the box. It is well suited for developers, as it allows you to leverage Git's version control capabilities while providing a straightforward way to publish your work online.

\paragraph{Creating a GitHub Account}
To get started with GitHub Pages, you'll need a GitHub account. If you don't have one already, follow these steps to create an account:
\begin{enumerate}
    \item Go to the \href{https://github.com/}{GitHub website}.
    \item Click on the ``Sign up'' button in the upper right corner.
    \item Fill out the registration form with your details and click ``Create account''.
    \item Verify your email address by clicking the link sent to your inbox.
    \item Once your account is created, you can log in to GitHub.
\end{enumerate}

\paragraph{Creating a Repository}
A repository is where your website's files will be stored on GitHub. To create a new repository for your website, follow these steps:
\begin{enumerate}
    \item Log in to your GitHub account \href{https://github.com/login}{here}.
    \item Click on the ``+'' icon in the upper right corner and select ``New repository''.
    \item Enter a name for your repository (e.g., ``my-website'') and add a description if desired.
    \item Leave the repository as ``Public'' as it is required for GitHub Pages to work for free hosting.
    \item Everything else can be left as default, but you can choose to initialize the repository with a README file if you wish.
    \item Click the ``Create repository'' button to finish.
\end{enumerate}

\subsection{Setting Up Your Local Development Environment}

Before you can start building your website, you'll need to set up your local development environment. This involves installing the necessary software and tools that will allow you to create and manage your website files effectively.

\subsubsection{Installing Git}
Git is a version control system that allows you to track changes in your code and collaborate with others. It is essential for managing your website's source code, especially when using GitHub Pages. To install Git, follow these steps:
\begin{enumerate}
    \item Go to the \href{https://git-scm.com/downloads}{Git website}.
    \item Download the appropriate version for your operating system (Windows, macOS, or Linux).
    \item Follow the installation instructions for your platform. During installation, you can choose the default options unless you have specific preferences.
    \item Once installed, open a terminal or command prompt and verify the installation by typing \texttt{git --version}. You should see the installed version of Git displayed.
\end{enumerate}

\subsubsection{Installing a Code Editor}
A code editor is a software application that allows you to write and edit your website's code. For this guide, we recommend using Visual Studio Code (VS Code) due to its popularity and extensive features. To install VS Code, follow these steps:
\begin{enumerate}
    \item Go to the \href{https://code.visualstudio.com/}{Visual Studio Code website}.
    \item Download the version for your operating system (Windows, macOS, or Linux).
    \item Follow the installation instructions for your platform.
    \item Once installed, open VS Code and familiarize yourself with its interface. You can customize the settings and install extensions to enhance your coding experience.
\end{enumerate}

