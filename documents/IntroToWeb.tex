
\newpage
\section{How Static Websites Work}

\subsection{Definitions and Concepts}
\begin{definition}{Static Website}
A static website is a type of website that delivers the same content to every user. It is built using HTML, CSS, and JavaScript, and the files are served directly to the user's browser without any server-side processing. Static websites are typically faster and more secure than dynamic websites, as they do not rely on databases or server-side scripts.
\end{definition}

\begin{definition}{HTML}
HTML (HyperText Markup Language) is the standard markup \\ language used to create the structure and content of web pages. It defines elements such as headings, paragraphs, links, images, and other multimedia content. HTML files are static and do not change unless manually edited.% \\ is to make sure the text is not too long
\end{definition}

\begin{definition}{CSS}
CSS (Cascading Style Sheets) is a stylesheet language used to describe the presentation of HTML documents. It allows you to control the layout, colors, fonts, and overall appearance of your website. CSS files are also static and are linked to HTML files to apply styles.
\end{definition}

\begin{definition}{JavaScript}
JavaScript is a programming language that enables interactive features on web pages. It can manipulate HTML and CSS, respond to user actions, and perform calculations. While JavaScript can be used in static websites, it is often associated with dynamic websites where server-side processing is involved.
\end{definition}

\begin{concept}{Local Development Versus Live Website}
When developing a static website, you typically work on your local machine using a code editor. This allows you to create and test your website files before deploying them to a live server. Once you are satisfied with your local version, you can upload the files to a hosting service, making them accessible to users on the internet.
\end{concept}

\subsection{How Static Websites Work}
Static websites work by serving pre-built HTML, CSS, and JavaScript files directly to the user's browser. When a user requests a static web page, the server retrieves the corresponding HTML file and sends it to the browser.\\ The browser then interprets the HTML, applies any linked CSS styles, and executes any JavaScript code to render the page. This process is straightforward and does not require any server-side processing, making static websites fast and efficient.

\vspace*{1cm}

\noindent\hspace*{0cm}
\begin{tikzpicture}[
    node distance=3cm and 3cm,
    every node/.style={font=\small\sf, align=center},
    box/.style={draw=blue!50!black, fill=blue!10, thick, rounded corners, minimum width=3.2cm, minimum height=1.2cm},
    filebox/.style={draw=orange!60!black, fill=orange!10, thick, rounded corners, minimum width=3.2cm, minimum height=1.2cm},
    arrow/.style={->, very thick, >=stealth}
]

% Nodes
\node[box] (user) {User\\(Web Browser)};
\node[box, right=of user] (internet) {Internet};
\node[box, right=of internet] (server) {Web Server\\(Hosting Provider)};
\node[filebox, below=of server] (files) {Static Files:\\HTML, CSS, JS};

% Arrows
\draw[arrow] 
  ([yshift=5pt]user.east) -- node[above, yshift=2pt]{1. Request Page} 
  ([yshift=5pt]internet.west);

\draw[arrow] 
  ([yshift=-5pt]internet.west) -- node[below, yshift=-2pt]{5. Deliver Page} 
  ([yshift=-5pt]user.east);

\draw[arrow] (internet) -- node[above]{2. Forward Request} (server);
\draw[arrow] (server) -- node[right]{3. Fetch Files} (files);

\draw[arrow] 
  (files.west) 
  to[out=180, in=-90] 
  node[midway, below left, font=\small\sf] {4. Send Files}
  ($(internet.south)+(0,-1.5)$) 
  to[out=90, in=270] 
  (internet.south);

\end{tikzpicture}

\vspace*{1cm}

This diagram illustrates the flow of a static website request:
\begin{enumerate}
    \item The user opens a web browser and requests a specific page by entering a URL.\ 
    \item The request is sent over the internet to the web server hosting the static files.
    \item The server retrieves the requested HTML file along with any associated CSS and JavaScript files.
    \item The server sends these files back to the user's browser.
    \item The browser renders the page, applying styles and executing scripts as needed.
\end{enumerate}

\subsection{Making your First Static Website}

Before you can create your first static website, you need to understand some basic concepts. Below is a simple HTML structure that you can use as a starting point for your static website. This example includes the essential elements of an HTML document, such as the doctype declaration, head section, and body content. I will also provide a brief explanation of each part of the code.

\begin{lstlisting}[language=html, caption=Basic HTML Structure]
<!DOCTYPE html>
<html lang="en">
<head>
    <meta charset="UTF-8">
    <meta name="viewport" content="width=device-width, initial-scale=1.0">
    <title>My First Static Website</title>
    <link rel="stylesheet" href="styles.css">
</head>
<body>
    <header>
        <h1>Welcome to My First Static Website</h1>
    </header>
    <main>
        <p>This is a simple static website created using HTML and CSS.</p>
    </main>
    <footer>
        <p>&copy; 2023 My First Static Website</p>
    </footer>
</body>
</html>  
\end{lstlisting}

\subsubsection{Explanation of the Code}

Notice that all the lines so far start with a \textit{<} and end with a \textit{>}. This is a fundamental characteristic of HTML, where elements are defined by tags. 

\begin{concept}{Tags in HTML}
Tags are the building blocks of HTML.\ They are used to define elements on a web page. Each tag typically consists of an opening tag (e.g., \texttt{<tagname>}) and a closing tag (e.g., \texttt{</tagname>}). The content between the opening and closing tags is what will be displayed on the web page. Some tags, like \texttt{<img>}, are self-closing and do not require a closing tag.\ example: \texttt{<img src="image.jpg" alt="Description">} where \texttt{src} is the source of the image and \texttt{alt} is the alternative text that describes the image. % chktex 18
\end{concept}
