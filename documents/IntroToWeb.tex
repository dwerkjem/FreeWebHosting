\newpage
\section{How Static Websites Work}

\subsection{Definitions and Concepts}
\begin{definition}[Static website]
A static website is a type of website that delivers the same content to every user. It is built using HTML, CSS, and JavaScript, and the files are served directly to the user's browser without any server-side processing. Static websites are typically faster and more secure than dynamic websites, as they do not rely on databases or server-side scripts.
\end{definition}

\begin{definition}[HTML]
HTML (HyperText Markup Language) is the standard markup \\ language used to create the structure and content of web pages. It defines elements such as headings, paragraphs, links, images, and other multimedia content. HTML files are static and do not change unless manually edited.% \\ is to make sure the text is not too long
\end{definition}

\begin{definition}[CSS]
CSS (Cascading Style Sheets) is a stylesheet language used to describe the presentation of HTML documents. It allows you to control the layout, colors, fonts, and overall appearance of your website. CSS files are also static and are linked to HTML files to apply styles.
\end{definition}

\begin{definition}[JavaScript]
JavaScript is a programming language that enables interactive features on web pages. It can manipulate HTML and CSS, respond to user actions, and perform calculations. While JavaScript can be used in static websites, it is often associated with dynamic websites where server-side processing is involved.
\end{definition}

\begin{concept}[Local Development Versus Live Website]
When developing a static website, you typically work on your local machine using a code editor. This allows you to create and test your website files before deploying them to a live server. Once you are satisfied with your local version, you can upload the files to a hosting service, making them accessible to users on the internet.
\end{concept}

\subsection{How Static Websites Work}
Static websites work by serving pre-built HTML, CSS, and JavaScript files directly to the user's browser. When a user requests a static web page, the server retrieves the corresponding HTML file and sends it to the browser.\\ The browser then interprets the HTML, applies any linked CSS styles, and executes any JavaScript code to render the page. This process is straightforward and does not require any server-side processing, making static websites fast and efficient.

